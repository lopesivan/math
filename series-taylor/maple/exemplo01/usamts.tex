\documentclass[11pt]{article}

%I am putting in brief descriptions along the way so this source file is not just a big clump of LaTeX coding
%include necessary packages and page settings

\usepackage{amsmath}

\pdfpagewidth 8.5in
\pdfpageheight 11in
\topmargin -1in
\headheight 0in
\headsep 0in
\textheight 8.5in
\textwidth 6.5in
\oddsidemargin 0in
\evensidemargin 0in
\headheight 77pt
\headsep 0in
\footskip .75in

\title{Solution for USAMTS Problem 4/4/15}
\author{Zachary Abel}

\begin{document}
\maketitle
%
%type up the question
%
\noindent\textbf{\large{Problem:}} For each non-negative integer $n$ define the function $f_n(x)$ by
\[
f_n(x)=\sin^n(x)+\sin^n\big(x+\frac{2\pi}{3}\big)+\sin^n\big(x+\frac{4\pi}{3}\big)
\]
for all real numbers $x$, where the sine fuctions use radians. The functions $f_n(x)$ can also be expressed as polynomials in $\sin(3x)$ with rational coefficients. For example,
\begin{gather*}
f_0(x)=3,\qquad f_1(x)=0,\qquad f_2(x)=\frac{3}{2},\qquad f_3(x)=-\frac{3}{4}\sin(3x),\\
f_4(x)=\frac{9}{8},\qquad f_5(x)=-\frac{15}{16}\sin(3x),\qquad f_6(x)=\frac{27}{32}+\frac{3}{16}\sin^2(3x),
\end{gather*}
for all real numbers $x$. Find an expression for $f_7(x)$ as a polynomial in $\sin(3x)$ with rational coefficients, and prove that it holds for all real numbers $x$.
\newline

\noindent \textbf{\large{Solution:}}
%
%make definitions
%
Define
\[
a=\sin(x),\qquad b=\sin\big(x+\frac{2\pi}{3}\big),\qquad c=\sin\big(x+\frac{4\pi}{3}\big),
\]
so that $f_n(x)=a^n+b^n+c^n$.
%
%find the value of a+b+c
%
First, we have
\[
a+b+c = f_1(x) = 0.
\]
\\
%
%find the value of ab+bc+ca
%
From the algebraic identity $(a+b+c)^2 = a^2+b^2+c^2+2(ab+bc+ca)$, we find that
\begin{gather*}
f_1(x)^2 = f_2(x)+2(ab+bc+ca)\\
ab+bc+ca = \frac{1}{2}\left(f_1(x)^2-f_2(x)\right)\\
ab+bc+ca = -\frac{3}{4}.
\end{gather*}
%
%find the value of abc
%
Finally, from the identity $a^3+b^3+c^3-3abc=(a+b+c)(a^2+b^2+c^2-ab-bc-ca)$,
\begin{gather*}
f_3(x)-3abc=f_1(x)\big(f_2(x)+\frac{3}{4}\big)\\
abc=\frac{1}{3}\left(f_3(x)-f_1(x)\big(f_2(x)+\frac{3}{4}\big)\right)\\
abc=-\frac{1}{4}\sin(3x).
\end{gather*}
%
%find the recursion for the function f
%
When $t$ equals $a$, $b$, or $c$, the quantity
\begin{align*}
F(t)&=(t-a)(t-b)(t-c)\\
&= t^3-(a+b+c)t^2+(ab+bc+ca)t-abc\\
&= t^3-\frac{3}{4}t+\frac{1}{4}\sin(3x)
\end{align*}
is identically 0. Specifically, we get the three following equations:
\begin{gather*}
a^n F(a)=a^{n+3}-\frac{3}{4}a^{n+1}+\frac{1}{4}a^n\sin(3x)=0\\
b^n\,F(b)=\ b^{n+3}-\frac{3}{4}b^{n+1}+\frac{1}{4}b^n\sin(3x)=0\\
c^n\,F(c)=\ c^{n+3}-\frac{3}{4}c^{n+1}+\frac{1}{4}c^n\sin(3x)=0
\end{gather*}
Adding the three previous equations gives a marvelous property of our function:
\begin{equation}\label{recurrence}
f_{n+3}(x)=\frac{3}{4}f_{n+1}(x)-\frac{1}{4}f_n(x)\sin(3x)
\end{equation}
%
%solve the problem
%
Using this recursion, the answer to the problem is immediate:
\begin{align*}
f_7(x)&=\frac{3}{4}f_5(x)-\frac{1}{4}f_4(x)\sin(3x)\\
&=-\frac{45}{64}\sin(3x)-\frac{9}{32}\sin(3x)\\
&=-\frac{63}{64}\sin(3x).
\end{align*}
%
%closing comments
%
Notice that the recursion given in equation (\ref{recurrence}) proves that $f_n(x)$ is a polynomial in $\sin(3x)$ with rational coefficients for all natural $n$.
\end{document}